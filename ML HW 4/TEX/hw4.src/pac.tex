\section{PAC Learning}
\label{sec:pac-learning}

\begin{enumerate}
\item A factory assembles a product that consist of different
  parts. Suppose a robot was invented to recognize whether a product
  contains all the right parts. The rules for making products are
  very simple: 1) you are free to combine any of the parts as they
  are 2) you may also cut any of the parts into two distinct pieces
  before using them.

  You wonder how much effort a robot would need to figure out the
  what parts are used in the product.

  \begin{itemize}
  \item\relax [5 points] Suppose that a naive robot has to recognize
    products made using only rule 1. Given $N$ available parts and
    each product made out of these constitutes a distinct
    hypothesis. How large would the hypothesis space be? Brief
    explain your answer.
  
  \item\relax [5 points] Suppose that an experienced worker follows
    both rules when making a product. How large is the hypothesis
    space now?  Explain.

  \item\relax [10 points] An experienced worker decides to train the
    naive robot to discern the makeup of a product by showing it the
    product samples he has assembled. There are 6 available
    parts. If the robot has to learn any product at $0.01$ error
    with probability $99\%$, how many examples would the robot have
    to see?
  \end{itemize}


  
\item \relax[20 points] Consider the class $C$ of concepts of the
  form $(a\leq x \leq b) \land (c \leq y \leq d)$ where $a,b,c,d$
  are integers in the interval $(0,20)$. Each concept in this class
  consists of a rectangle with integer valued boundaries and labels
  all points inside the rectangle as positive and everything outside
  as negative.

  Give an upper bound on the number of randomly drawn examples
  needed to ensure that for any function $c$ in the set $C$, a
  consistent learner that uses $H = C$ will, with probability $99\%$
  produce a classifier whose error is no more than $0.01$.

  Hint: To answer this question, you could use the fact that in a
  plane bounded by the points $(0,0)$ and $(n, n)$, the number of
  distinct rectangles with integer-valued boundaries in the region
  is $\left(\frac{n(n+1)}{2}\right)^2$.
  
\end{enumerate}



%%% Local Variables:
%%% mode: latex
%%% TeX-master: "hw"
%%% End:
