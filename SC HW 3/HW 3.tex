\documentclass[11pt,a4paper]{article}
\usepackage[latin1]{inputenc}
\usepackage[margin=1in]{geometry}
\usepackage{amsmath}
\usepackage{amsfonts}
\usepackage{amssymb}
\usepackage{graphicx}
\usepackage{enumitem}
\usepackage{listings}
\usepackage{color}

\definecolor{dkgreen}{rgb}{0,0.6,0}
\definecolor{gray}{rgb}{0.5,0.5,0.5}
\definecolor{mauve}{rgb}{0.58,0,0.82}

\lstset{frame=tb,
 language=MatLab,
 aboveskip=3mm,
 belowskip=3mm,
 showstringspaces=false,
 columns=flexible,
 basicstyle={\small\ttfamily},
 numbers=none,
 numberstyle=\tiny\color{gray},
 keywordstyle=\color{blue},
 commentstyle=\color{dkgreen},
 stringstyle=\color{mauve},
 breaklines=true,
 breakatwhitespace=true,
 tabsize=3
}

\setlength\abovedisplayskip{0pt}
\author{James Brissette}
\title{CS-6210: HW 3}
\begin{document}
	\maketitle
	
	\section{Chapter 5}
		\begin{itemize}
			\item[5.10]
				\begin{enumerate} [label={\alph*)}]
					\item ~
					\begin{lstlisting} 
					\end{lstlisting}
					\item ~
					
					\item ~
					\item ~
				\end{enumerate}
					
			\item[5.11]
				\begin{enumerate} [label={\alph*)}]
					\item For $g(x)$ to be a cubic spline, the following basic conditions must be met:
					$$\begin{array}{cc}
						g_1(x_1)=y1 & g_1(x_2)=y2 \\
						g_2(x_2)=y2 & g_2(x_3)=y3 
					\end{array}$$~
					$$\begin{array}{c}
						g'_1(x_2)=g'_2(x_2)\\
						g''_1(x_2)=g''_2(x_2)
					\end{array}$$
					Looking at the first derivatives
					\item
					\item
					\item 
				\end{enumerate}
				
			\item[5.15]
				\begin{enumerate} [label={\alph*)}]
					\item ~
				\end{enumerate}
				
			\item[5.22]
				\begin{enumerate} [label={\alph*)}]
					\item ~
				\end{enumerate}
				
			\item[5.26]
				\begin{enumerate} [label={\alph*)}]
					\item ~
				\end{enumerate}
				
			\item[5.27]
				\begin{enumerate} [label={\alph*)}]
					\item ~
				\end{enumerate}
				
			\item[5.28]
				\begin{enumerate} [label={\alph*)}]
					\item ~
				\end{enumerate}
		\end{itemize}
		
	
\end{document}