\documentclass[11pt,a4paper]{article}
\usepackage[latin1]{inputenc}
\usepackage[margin=1in]{geometry}
\usepackage{amsmath}
\usepackage{amsfonts}
\usepackage{amssymb}
\usepackage{graphicx}
\usepackage{enumitem}
\usepackage{booktabs}
\setlength\abovedisplayskip{0pt}
\author{James Brissette}
\title{CS-6190: Homework 1}
\begin{document}
	\maketitle
	
	\section{Warm Up}
		\begin{enumerate}
			\item To get $p(x_1)$ from $x = \begin{bmatrix} x_1 \\ x_2 \end{bmatrix}$ we need to use a fair bit of wizardry, and assume $\Sigma = \begin{bmatrix} \Sigma_{11} & \Sigma_{12} \\ \Sigma_{21} & \Sigma_{22}\end{bmatrix}$, so $\Sigma^{-1} = V = \begin{bmatrix} V_{11} & V_{12} \\ V_{21} & V_{22}\end{bmatrix}$. Furthermore, we know that $x^TAx = \sum_i \sum_j x_iA_{ij}x_j$, so the term in our multivariate gaussian exponent becomes: 
			\begin{align*}
				&\Big(\frac{1}{2\pi}\Big)^{\frac{k}{2}} \frac{1}{| \Sigma |^\frac{1}{2}} exp \big[ \frac{1}{2}x^T\Sigma^{-1}x \big] \\
				&= x_1^T V_{11} x_1 + x_1^T V_{12} x_2 + x_2^T V_{21} x_1 + x_2^T V_{22} x_2
			\end{align*}
			When we go through and tediously complete the square we get something that looks like the following:
			\begin{align*}
			x^T \Sigma^{-1}x = (x_2 + V_{22}^{-1}V_{21}x_1)^T V_{22}(X_2 + V_{22}^{-1}V_{21}x_1) +x_1^T(V_{11}-V_{21}^T V_{22}^{-1}V_{21})x_1
			\end{align*}
			and with some guidance from various sources that illustrate that $f(x) = f(x_{2}|x_{1})f(x_{1})$, we can simplify this and recover the marginal distribution of $x_1$ as follows:
			\begin{align*}
				f(x_{2} | x_{1}) &\propto exp(x_2 + V_{22}^{-1}V_{21}x_1)^T V_{22}(X_2 + V_{22}^{-1}V_{21}x_1) \\
				f(x_1) &\propto exp((x_1^T - \mu)^T(V_{11}-V_{21}^T V_{22}^{-1}V_{21})(x_1-\mu)) \\
			X_1 &\mathtt{\sim} \mathcal{N}(\mu, (V_{11}-V_{21}^T V_{22}^{-1}V_{21}))
			\end{align*}
			Which turns out to be exactly $X_1 \mathtt{\sim} \mathcal{N}(\mu, \Sigma_{11})$
			\item
			\item We can simplify this equation by using the definition of Expected value and some Matrix Cookbook trace tricks:
			\begin{align*}
			H(x) &= -\int p(x)log(p(x)) dx \\
			&= -\int \mathcal{N}(x|\mu,\Sigma)*log\mathcal{N}(x|\mu,\Sigma) \\
			&= -\mathbb{E}\Big(log\mathcal{N}(x|\mu,\Sigma)\Big) \\
			&= -\mathbb{E}\Big(log[2\pi^{\frac{d}{2}}|\Sigma|^{-\frac{1}{2}}exp\big(-\frac{1}{2}(x-\mu)^T \Sigma^{-1} (x-\mu)\big)]\Big)\\
			&= -\mathbb{E}(-\frac{d}{2}log(2\pi)) -\mathbb{E}(-\frac{1}{2}log|\Sigma|)-\mathbb{E}(-\frac{1}{2}(x-\mu)^T \Sigma^{-1} (x-\mu)) \\
			&= \frac{d}{2}log(2\pi) + \frac{1}{2}log|\Sigma| + \frac{1}{2}\mathbb{E}((x-\mu)^T \Sigma^{-1} (x-\mu)) \\
			&= \frac{d}{2}log(2\pi) + \frac{1}{2}log|\Sigma| + \frac{1}{2}\mathbb{E}(tr((x-\mu)^T \Sigma^{-1} (x-\mu))) \\
			&= \frac{d}{2}log(2\pi) + \frac{1}{2}log|\Sigma| + \frac{1}{2}\mathbb{E}(tr(\Sigma^{-1} (x-\mu)(x-\mu)^T )) \\
			&= \frac{d}{2}log(2\pi) + \frac{1}{2}log|\Sigma| + \frac{1}{2}tr(\mathbb{E}(\Sigma^{-1} (x-\mu)(x-\mu)^T )) \\
			&= \frac{d}{2}log(2\pi) + \frac{1}{2}log|\Sigma| + \frac{1}{2}tr(\Sigma^{-1}\mathbb{E}( (x-\mu)(x-\mu)^T )) \\
			&= \frac{d}{2}log(2\pi) + \frac{1}{2}log|\Sigma| + \frac{1}{2}tr(\Sigma^{-1}\Sigma) \\
			&= \frac{d}{2}log(2\pi) + \frac{1}{2}log|\Sigma| + \frac{1}{2}tr(I_d) \\
			&= \frac{d}{2}log(2\pi) + \frac{1}{2}log|\Sigma| + \frac{d}{2} \\
			&= \frac{d}{2}(1+log(2\pi)) + \frac{1}{2}log|\Sigma|
\end{align*}						
			
			\item We know that $KL(q||p) = \int \big[ log(q(x)) - log(p(x)) \big] p(x) dx$. If we expand out each term we get:
			\begin{align*}
				&\int log \Big[ \frac{1}{2| \Lambda |}exp(\frac{1}{2}(x-m)^T \Lambda^{-1}(x-m)) \Big] - log \Big[ \frac{1}{2| \Sigma |}exp(\frac{1}{2}(x-\mu)^T \Sigma^{-1}(x-\mu)) \Big] p(x)dx \\
				&= \int \Big[ \frac{1}{2} log \frac{| \Lambda |}{| \Sigma |} - \frac{1}{2}(x-m)^T \Lambda^{-1} (x-m) + \frac{1}{2}(x-\mu)^T \Sigma^{-1} (x-\mu) \Big] p(x) dx \\
				&= \frac{1}{2} log \frac{| \Lambda |}{| \Sigma | }-tr\big[ E[(x-\mu)(x-\mu)^T] \Sigma^{-1}\big] + \frac{1}{2} E[(x-m)^T \Lambda^{-1}(x-m)]
			\end{align*}
			and we know from the matrix cookbook that $E(x^TAx) = tr(A\Sigma)+m^TAm$:
			\begin{align*}
				&= \frac{1}{2} log \frac{| \Lambda |}{| \Sigma | }-tr\big[ I_d \big] + \frac{1}{2} (m-\mu) + \frac{1}{2}tr(\Lamda^{-1}\Sigma)) \\
				&= \frac{1}{2} \big[log \frac{| \Lambda |}{| \Sigma | } - d + tr(\Lambda^{-1}\Sigma) + (m- \mu)^T \Lambda^{-1}(m- \mu)\big]
			\end{align*}
			\item So if we use a fair bit of what is in the lecture notes, this isn't too bad. We know that $\nabla log(Z(\eta)) = \frac{\nabla Z(\eta)}{Z(\eta)}$, that $Z(\eta)= \int h(x)exp(u(x)^T \eta)dx$ and that $\nabla Z(\eta) = \int h(x) exp(u(x)^T\eta) u(x) dx$.
			
			If we combine terms to simplify the gradient by taking $\frac{1}{Z(\eta)}$ to be $exp(-log(Z(\eta)))$ we can combine terms and calculate the second derivative as follows:
			\begin{align*}
				\nabla log(Z(\eta)) &= \int u(x) h(x) exp(u(x)^T\eta - log(Z(\eta))) \\
				\nabla^2 \frac{log(Z(\eta))}{d\eta} &= \int u(x) h(x) exp(u(x)^T\eta - log(Z(\eta)))(u(x)-\nabla log(Z(\eta))
			\end{align*}
			
			And we can simplify $u(x)exp(u(x)^T\eta - log(Z(\eta))) = \mathbb{E}[u(x)]$ which gives us:
			
			\begin{align*}
			&= \int u(x)^2 exp(u(x)^T\eta - log(Z(\eta)) - u(x)\mathbb{E}[u(x)]exp(u(x)^T\eta - log(Z(\eta))) \\
			&= \mathbb{E}[u(x)^2] - \mathbb{E}[u(x)]^2 \\
			&= \mathbb{V}[u(x)]
			\end{align*}
			
			\item I've never heard of negative Variance, so the covariance matrix must all be positive, meaning it's positive semi-definite, meaning it's convex.
			\item In order to calculate the mutual information in terms of the entropy of two random variables, we'll start with the definition for Mutual Information:
			\begin{align*}
				I &= -\int \int p(x,y) log \frac{p(x)p(y)}{p(x,y)} \\
				&= -\int \int p(x,y) \Big[ log \frac{p(y)}{p(x,y)} + log(p(x)) \Big] \\
				&= -\int \int p(x,y) * log \frac{p(y)}{p(x,y)} + \int \int p(x,y) log(p(x)) \\
				&= -\int \int p(y) p(x|y) * log (p(x|y)) + \int \int p(x,y) log(p(x)) \\
				&= -\int p(y) \int p(x|y) * log (p(x|y)) + \int log(p(x)) \int p(x,y)  \\
				&= -\int p(y) * H(x|y) + \int log(p(x)) * p(x)  \\
				&= -\int p(y) * H(x|y) \quad  + H(x)  \\
				&= -H(x|y) + H(x)  \\
				&= H(x) - H(x|y)
			\end{align*}
			\item 
			\begin{enumerate}
				\item Dirichlet: \\
				\begin{align*}
					&= \frac{\Gamma(\alpha_0)}{\Gamma(\alpha_1)\dots \Gamma(\alpha_k)}	\prod_{k=1}^K \mu_k^{a_k-1}	\\
					&= log(\Gamma(\alpha_0)) -\dots + \sum_{k=1}^K (a_k -1)(log(\mu_k)) \\
					&= log(\Gamma(\alpha_0)) -\dots + (a_k -1)^T(log(\mu_k)) \\
					&= exp(\eta^T T(x) + log(\frac{\Gamma(\alpha_0)}{\Gamma(\alpha_1)\dots \Gamma(\alpha_k)}))
				\end{align*}
				where $\eta = (\vec{a}+\vec{1})$, $T(x) = log(\vec{\mu})$ and $log(Z(\eta)) = log(\frac{\Gamma(\alpha_0)}{\Gamma(\alpha_1)\dots \Gamma(\alpha_k)})$
				\item Gamma: \\
				\begin{align*}
					Gam(\lambda |a,b) &= \frac{1}{\Gamma(a)}b^a\lambda^{a-1}exp(-b\lambda) \\
					&= exp(log(\frac{1}{\Gamma(a)}b^a\lambda^{a-1}exp(-b\lambda)))
				\end{align*}
				After taking the log, we have an idea that based on the function values, we have $\eta(\Theta) = \frac{1}{\Gamma(a)}$, $T(x) = b^a \lambda^{a-1}$. After simplifying and rearranging terms we get:
				\begin{align*}
					exp(-b\lambda + a*log(b) + (a-1)log(\lambda) - log(\Gamma(a)))
				\end{align*}
				from which we can deduce that $\eta(a,\lambda) = (-b,a)^T$ and $T(b) = (b, log(b))^T$
				\item
			\end{enumerate}
			\item
			\item Yes.
			\item 
			\begin{enumerate}
				\item 
				\item
				\item
				\item
			\end{enumerate}
					
		\end{enumerate}
\end{document}