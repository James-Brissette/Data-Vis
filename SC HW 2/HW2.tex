\documentclass[11pt,a4paper]{article}
\usepackage[latin1]{inputenc}
\usepackage[margin=1in]{geometry}
\usepackage{amsmath}
\usepackage{amsfonts}
\usepackage{amssymb}
\usepackage{graphicx}
\usepackage{enumitem}

\setlength\abovedisplayskip{0pt}
\author{James Brissette}
\title{CS-6210: HW 2}
\begin{document}
	\maketitle
	
	\section{Chapter 4}
		\begin{itemize}
			\item[4.22]
				\begin{enumerate} [label={\alph*)}]
					\item 
					\item My approximations of 1 were off by 4.50099e-08 in either direction.
					\item
					\item
				\end{enumerate}
					
			\item[4.31]
				\begin{enumerate} [label={\alph*)}]
					\item 
					\item
					\item
					\item
				\end{enumerate}
		\end{itemize}
		
	\section{Chapter 9}
		\begin{itemize}
			\item[9.2]
				\begin{enumerate} [label={\alph*)}]
					\item The result of the normalization using $\overline{X}=2.8215$, $\overline{Y}=2.4224$, and scaling $X$ by $7.1118$ and $Y$ by $2.0859$ is:
						$$\begin{array}{cc}
							p & q \\ \hline
							-1.0000  & 0.5963 \\
							-0.8151  & 0.6407 \\
							-0.6096  & 0.2867 \\
							-0.3397  & 0.5540 \\
							-0.1761  & 0.4480 \\
							0.2006 & -0.1704 \\
							0.2508  & 0.0572 \\
							0.4053   & -0.0117 \\
							0.5854  & -0.5213 \\
							0.6720  & -0.8795 \\
							0.8265  & -1.0000 \\
							
						
						\end{array}$$
					\item The $\alpha$ calculated using the normalized table in part a: $$\alpha = -0.9281$$
					\item Since we have $G(X)=V_1 + V_2 X$ where $G(X)$ is a linear function with slope $V_2$, it follows that for the analogous function $Y$,
					$$Y=\overline{Y} + m(X-\overline{X})$$
					Rearranging Y we group the constants and get:
					$$Y=(\overline{Y}-m\overline{X})+mX$$ Since $V_1=log(v_1)$, we can solve for $v_1$ as:
					$$V_1= log(v_1)=\overline{Y}-m\overline{X}$$
					$$v_1=10^{\overline{Y}-m\overline{X}}$$
					\item Plot Curve
					\item Estimate speed of T-Rex
				\end{enumerate}
					
			\item[9.3]
				\begin{enumerate} [label={\alph*)}]
					\item Given $ZZ^T = \frac{1}{n} \Sigma_{i=1}^{n} s_i(s_i)^T$ and $s_i^* = Z^{-1}s_i$ we can plug into 9.30 and get back the original $ZZ^T$ as follows:
					\begin{equation}
						\frac{1}{n} \Sigma_{i=1}^{n} s_i^*(s_i^*)^T = I
					\end{equation}
					\begin{equation}
						\frac{1}{n} \Sigma_{i=1}^{n} Z^{-1}s_i(Z^{-1}s_i)^T = I
					\end{equation}
					\begin{equation}
						\frac{1}{n} \Sigma_{i=1}^{n} Z^{-1}s_i (s_i)^T Z^{-1T}= I
					\end{equation}
					\begin{equation}
						Z^{-1} (\frac{1}{n} \Sigma_{i=1}^{n} s_i (s_i)^T) Z^{-1T}= I
					\end{equation}
					\begin{equation}
						ZZ^{-1} (\frac{1}{n} \Sigma_{i=1}^{n} s_i (s_i)^T) Z^{-1T}Z^T= ZIZ^T
					\end{equation}
					\begin{equation}
						\frac{1}{n} \Sigma_{i=1}^{n} s_i(s_i)^T = ZZ^T
					\end{equation}
					
					\setcounter{equation}{0}
					\item For this problem we will show that given $\frac{1}{n} \Sigma_{i=1}^{n} s_i(s_i)^T = U \Sigma U^T$ and $s_{i}^{*} = \Sigma^{-\frac{1}{2}} U^T s_i$ we can plug $s_i^*$ into 9.30 and get back the identity matrix $I$:
						\begin{equation}
							\frac{1}{n} \Sigma_{i=1}^{n} s_i^*(s_i^*)^T = I
						\end{equation}
						\begin{equation}
							\frac{1}{n} \Sigma_{i=1}^{n} (\Sigma^{-\frac{1}{2}} U^T s_i)(\Sigma^{-\frac{1}{2}} U^T s_i)^T = I
						\end{equation}
						\begin{equation}
							\frac{1}{n} \Sigma_{i=1}^{n} (\Sigma^{-\frac{1}{2}} U^T s_i)(s_i^T  U^{T^T}  \Sigma^{-\frac{1}{2}T}) = I
						\end{equation}
						\begin{equation}
							\Sigma^{-\frac{1}{2}}U^T (\frac{1}{n} \Sigma_{i=1}^{n} s_i(s_i^T))U\Sigma^{-\frac{1}{2}T} = I
						\end{equation}
						\begin{equation}
							\Sigma^{-\frac{1}{2}}U^T (U \Sigma U^T) U\Sigma^{-\frac{1}{2}T} = I
						\end{equation}
						\begin{equation}
							\Sigma^{-\frac{1}{2}}(\Sigma )\Sigma^{-\frac{1}{2}T} = I
						\end{equation}
						And since we know $\Sigma$ is a diagonal matrix:
						\begin{equation}
							I = I
						\end{equation}
						
 				\end{enumerate}
				
			\item[9.5]
				\begin{enumerate} [label={\alph*)}]
					\item
				\end{enumerate}
				
			\item[9.8]
				\begin{enumerate} [label={\alph*)}]
					\item
				\end{enumerate}
		\end{itemize}
		
	
\end{document}